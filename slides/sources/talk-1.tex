\documentclass[10pt]{beamer}

\usetheme{metropolis}
\usepackage{appendixnumberbeamer}

\usepackage{booktabs}
\usepackage[scale=2]{ccicons}

\usepackage{pgfplots}
\usepgfplotslibrary{dateplot}

\usepackage{xspace}
\newcommand{\themename}{\textbf{\textsc{metropolis}}\xspace}

\title{Understanding the PhD Landscape in Europe and Germany}
\subtitle{Talk 1: Introduction and Getting Started}
\date{\today}
\author{Dr. Bijun Li}
%\institute{Center for modern beamer themes}
% \titlegraphic{\hfill\includegraphics[height=1.5cm]{logo.pdf}}

\begin{document}

\maketitle

%\section{Introduction}

% new page 
\begin{frame}[fragile]{Self-Introduction}


\end{frame}

% new page
\begin{frame}[fragile]{Purpose of the Series}

\alert{Explore the Opportunity of Pursuing a PhD in Europe}
\begin{itemize}
	\item Provide an overview of the PhD landscape in Europe.
	\item Focus on Germany's highly regarded PhD programs.
	\item Provide detailed information on the application process, funding opportunities, and practical considerations specific to Germany.
\end{itemize}

\alert{Target Audience}
\begin{itemize}
	\item Graduate school students currently pursuing a Master's degree and considering a PhD.
	\item Students in the final year of their undergraduate studies exploring future academic opportunities and seeking information on how to prepare for a PhD application in Europe from an early stage.
\end{itemize}
  
\end{frame}

% new page
\begin{frame}[fragile]{Overview}
  %\setbeamertemplate{section in toc}[sections numbered]
  %\tableofcontents[hideallsubsections]
\alert{Content to be Covered:}
  \begin{itemize}
    \item \textbf{Introduction to PhD Programs in Europe}
    \item \textbf{Understanding the German Education System}
    \item \textbf{Finding PhD Opportunities and Supervisors}
    \item \textbf{Application Process and Requirements}
  \end{itemize}
\end{frame}

% new page
\begin{frame}[fragile]{PhD Programs in Europe}
\alert{General Structure}
\begin{itemize}
	\item PhD programs in Europe generally last between 3 to 4 years.
	\item Requirements typically include a Master’s degree or equivalent qualification.
	\item PhD candidates must complete original research culminating in a dissertation or thesis.
	\item Traditional PhD: independent research, minimal coursework, close supervision by a single academic advisor.
	\item Structured PhD: includes formal coursework and seminars, collaboration with research groups, multiple supervisors or a supervisory committee, regular progress reviews and milestones.
\end{itemize}
Comparative Analysis:
Differences between European countries.

\end{frame}

% new page
\begin{frame}[fragile]{PhD Programs in Europe (cont.)}
\alert{Comparative Analysis}
\begin{itemize}
	\item Germany
	\begin{itemize}
		\item Strong focus on research universities and applied sciences.
		\item Well-established industry collaborations.
		\item Typically involves a working contract with a salary.
	\end{itemize}
	\item France
	\begin{itemize}
		\item PhD programs often linked to specific research institutions (CNRS).
		\item Structured programs with defined coursework.
	\end{itemize}
	\item Netherlands
	\begin{itemize}
		\item Known for a structured approach with clear timelines.
		\item Often involves teaching responsibilities.
	\end{itemize}
	\item Scandinavia (e.g., Sweden, Denmark)
	\begin{itemize}
		\item Emphasis on research projects with societal impact.
		\item Programs may include mandatory teaching and public engagement activities.
	\end{itemize}
\end{itemize}

\end{frame}


%\section{Title formats}
% new page
\begin{frame}[fragile]{Overview of the German Higher Education System}
\alert{Research Universities (Universitäten)}
	\begin{itemize}
		\item Focus on academic research and theoretical knowledge.
		\item Known for extensive research opportunities and doctoral programs.
		\item Examples: TU Munich, Heidelberg University, University of Freiburg.
	\end{itemize}
\alert{Universities of Applied Sciences (Fachhochschulen)}
	\begin{itemize}
		\item Strong ties with industry and business sectors.
		\item Offer practice-oriented programs and applied research.
		\item Examples: Munich University of Applied Sciences (Hochschule München), Berlin School of Economics and Law. 
	\end{itemize}
\alert{Research Institutes}
	\begin{itemize}
		\item Dedicated to advanced research in specific fields.
		\item Often operate independently or in affiliation with universities.
		\item Examples: Max Planck Society, Fraunhofer Institutes, Helmholtz Association.
	\end{itemize}
\end{frame}

% new page
\begin{frame}[fragile]{Overview of the German Higher Education System (cont.)}
\alert{Research and Industry Collaborations}
	\begin{itemize}
		\item Enhance Practical Experience: Collaborations with industry provide PhD students with hands-on experience and exposure to real-world applications of their research.
		\item Funding Opportunities: Industry partnerships often come with funding for research projects, scholarships, and grants, reducing financial burdens for students.
		\item Career Prospects: Working with industry partners helps build professional networks and improves employment opportunities post-PhD.
		\item Access to Resources: Collaboration with industry gives students access to advanced technologies, facilities, and expertise not always available within academic settings.
	\end{itemize}
\end{frame}

% new page
\begin{frame}[fragile]{Finding PhD Opportunities}
\alert{Researching PhD Programs}

	
\end{frame}

% new page
\begin{frame}[fragile]{Finding PhD Opportunities (cont.)}
\alert{Contacting Potential Supervisors}
	\begin{itemize}
		\item Identifying Potential Supervisors:
		\begin{itemize}
			\item Look for researchers whose work aligns with your interests.
			\item Read their recent publications to understand their research focus and methodologies.
			\item Check their profiles on university websites and research networks.
		\end{itemize}
		\item Approaching Potential Supervisors:
		\begin{itemize}
			\item Draft a concise and professional email introducing yourself and your research interests.
			\item Mention specific aspects of their work that you find compelling.
			\item Attach your CV, academic transcripts, and a brief research proposal if applicable.
		\end{itemize}
		\item Preparing a Compelling Research Proposal:
		\begin{itemize}
			\item Clearly define your research question or hypothesis.
			\item Outline your research objectives and methodology.
			\item Explain the significance of your research and its potential contributions to the field.
			\item Tailor your proposal to align with the supervisor’s current research projects.
		\end{itemize}
	\end{itemize}
	
\end{frame}

% new page
\begin{frame}[fragile]{Finding PhD Opportunities (cont.)}
\alert{Networking}
	\begin{itemize}
		\item Attending Conferences:
		\begin{itemize}
			\item Participate in local, national, and international conferences relevant to your field.
			\item Present your research or attend sessions to learn about the latest developments.
		\end{itemize}
		\item Workshops:
		\begin{itemize}
			\item Join workshops and seminars to enhance your skills and knowledge.
			\item Engage with speakers and other participants during networking sessions.
		\end{itemize}
		\item Academic Events:
		\begin{itemize}
			\item Take part in university-hosted events, guest lectures, and research symposiums.
			\item Join academic societies and professional organizations to expand your network.
		\end{itemize}
	\end{itemize}
	
\end{frame}


% new page
\begin{frame}[fragile]{Application Process and Requirements}
\alert{Eligibility Criteria}
\begin{itemize}
	\item Academic Qualifications: 
	\begin{itemize}
		\item Generally, a Master’s degree or equivalent is required.
		\item Relevant field of study and prior research experience can be crucial.
		\item Check specific requirements of each university or program.
	\end{itemize}
	\item Language Proficiency:
	\begin{itemize}
		\item English C1 (IELTS 7, TOEFL iBT 100)
		\item For programs in German, tests like TestDaF or DSH might be required.
	\end{itemize}
\end{itemize}
\end{frame}

% new page
\begin{frame}[fragile]{Application Process and Requirements (cont.)}
\alert{Application Procedures}
\begin{itemize}
	\item Required Documents:
	\begin{itemize}
		\item Curriculum Vitae (CV): Detailed overview of academic and professional background. Include research experience, publications, and relevant skills.
		\item Academic Transcripts: Official records of all academic qualifications.
Certified translations may be required if documents are not in English or German.
		\item Letters of Recommendation: Typically 2-3 letters from professors or professionals familiar with your work.
Should highlight your academic abilities, research potential, and character.
		\item Research Proposal: Outline of your proposed research project.
Should include research questions, methodology, literature review, and potential contributions to the field.
		\item Statement of Purpose (optional in some programs): Personal essay explaining your motivation for pursuing a PhD and choosing the particular program or university.
		\item Proof of Language Proficiency (optional)
	\end{itemize}
\end{itemize}
\end{frame}

% new page
\begin{frame}[fragile]{Application Process and Requirements (cont.)}
\alert{Timeline}
\begin{itemize}
	\item Key Deadlines and Milestones:
	\begin{itemize}
		\item Research and Identify Programs: Start 12-18 months before desired start date.
		\item Contact Potential Supervisors:Begin reaching out 9-12 months in advance.
Discuss your research proposal and seek their agreement to supervise your PhD.
		\item Prepare Application Materials: Gather documents, write your research proposal, and request recommendation letters 6-9 months ahead.
Should highlight your academic abilities, research potential, and character.
		\item Submit Applications: Application deadlines vary; typically fall between November and January for programs starting in the fall.
Some programs have rolling admissions or specific deadlines in the spring.
		\item Interviews and Decisions: Interviews (if required) usually take place 2-4 months after the application deadline.
Admission decisions are typically communicated 1-2 months after interviews.
	\end{itemize}
\end{itemize}
\end{frame}

% new page
\begin{frame}[fragile]{Conclusion and Preview of Talk 2}
	Summary of Key Points:
	Next Talk Preview: Focus on funding, financial aspects, practical considerations, and challenges
\end{frame}


\end{document}
