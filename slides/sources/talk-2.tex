\documentclass[10pt]{beamer}

\usetheme{metropolis}
\usepackage{appendixnumberbeamer}

\usepackage{booktabs}
\usepackage[scale=2]{ccicons}

\usepackage{pgfplots}
\usepgfplotslibrary{dateplot}

\usepackage{xspace}
\newcommand{\themename}{\textbf{\textsc{metropolis}}\xspace}

\title{Understanding the PhD Landscape in Europe and Germany}
\subtitle{Talk 2: Funding, Practical Considerations, and Challenges}
\date{\today}
\author{Dr. Bijun Li}
%\institute{Center for modern beamer themes}
% \titlegraphic{\hfill\includegraphics[height=1.5cm]{logo.pdf}}

\begin{document}

\maketitle

%\section{Introduction}

% new page 
\begin{frame}[fragile]{Introduction}
\alert{Recap from Talk 1}
\begin{itemize}
	\item Introduction to PhD programs in Europe and Germany.
	\item Overview of the German higher education system.
	\item How to find PhD opportunities and supervisors.
	\item Application process and requirements.
	\item Importance of language proficiency and preparation of necessary documents.
\end{itemize}

\alert{What We Will Cover Today}
\begin{itemize}
	\item Funding Sources
	\item Financial Aspects
	\item Practical Considerations
	\item Challenges and Solutions
	\item Networking and Professional Development
\end{itemize}
\end{frame}

% new page
\begin{frame}[fragile]{Overview of Funding Sources}
\alert{Government Grants and Scholarships}
\begin{itemize}
	\item DAAD (German Academic Exchange Service): DAAD Research Grants, DAAD Scholarships for Developing Countries.
	\item European Commission Grants: Marie Skłodowska-Curie Actions (MSCA), Erasmus+ Program.
	\item National Funding Agencies: DFG (German Research Foundation), CNRS (France), NWO (Netherlands)
	\item CSC
\end{itemize}
\alert{University Funding}
\begin{itemize}
	\item Internal Scholarships
	\item Research Assistant Positions
\end{itemize}
\end{frame}

% new page
\begin{frame}[fragile]{Overview of Funding Sources (cont.)}
\alert{Industry Partnerships}
\begin{itemize}
	\item Collaborative Research Projects: Industrial PhD programs, joint research labs
	\item Company Scholarships and Grants
\end{itemize}
\alert{External Funding}
\begin{itemize}
	\item Horizon 2020: Major EU funding program for research and innovation.
	\item Horizon Europe: Successor to Horizon 2020, continuing support for research initiatives.
\end{itemize}
\end{frame}

% new page
\begin{frame}{Financial Aspects}
\alert{Cost of Living (Estimated Expenses)}
  \begin{itemize}
  	\item Rent: €300-€800 per month depending on location and type of accommodation.
	\item Food: €200-€400 per month for groceries and dining.
	\item Transportation: €50-€100 per month for public transport (discounts for students).
	\item Other expenses: Internet and Utilities: €50 - €100, leisure and Entertainment: €50 - €100
  \end{itemize}
\end{frame}

% new page
\begin{frame}{Financial Aspects (cont.)}
\alert{PhD Salary}
\begin{itemize}
	\item German Researcher position TV-L13: 
	\begin{itemize}
		\item 100\% contract: €4200 - €6000 (gross) depending on years of experience
		\item 75\% contract: €3150 - €4500 (gross) depending on years of experience
	\end{itemize}
	\item Benefits included:
	\begin{itemize}
		\item Health Insurance: Coverage provided
		\item Pension Contributions: Included
		\item Paid Leave: Typically 25 - 30 days per year
	\end{itemize}
	\item Net salary: 100\% contract: €2600 - €3560
\end{itemize}
\end{frame}

% new page
\begin{frame}[fragile]{Practical Considerations}
\alert{Work-Life Balance}
\begin{itemize}
	\item Managing Academic Responsibilities
	\item Schedule dedicated time for research and work
	\item Set realistic goals and monitor progress
\end{itemize}
\alert{Cultural and Social Aspects}
\begin{itemize}
	\item Integrating into German society: local events and activities, student organizations and clubs to build social network.
	\item Language considerations and learning German: language courses offered by universities.
\end{itemize}
\alert{Health and Well-being}
\begin{itemize}
	\item Access to healthcare, insurance, and support services: register with a local general practitioner (Hausarzt) for regular check-ups. Utilize university counseling services for mental health support.
\end{itemize}
\end{frame}


% new page
\begin{frame}[fragile]{Challenges and Solutions}
\alert{Common Challenges}
	\begin{itemize}
		\item Balancing work and research
		\begin{itemize}
			\item Time Management: Juggling job responsibilities with research can be demanding.
			\item Burnout Risk: High workload may lead to stress and burnout.
			\item Prioritization: Difficulty in determining which tasks to prioritize.
		\end{itemize}
		\item Administrative Processes and Bureaucracy
		\begin{itemize}
			\item Complex Paperwork: Handling visa applications, permits, and other documentation.
			\item Language Barrier: Administrative documents may be in German.
		\item University Bureaucracy: Dealing with university-specific administrative requirements.
		\end{itemize}
	\end{itemize}

\end{frame}

% new page
\begin{frame}[fragile]{Challenges and Solutions}
\alert{Potential Solutions and Tips}
	\begin{itemize}
		\item Effective Communication with Supervisors:
		\begin{itemize}
			\item Regular Meetings: Schedule frequent check-ins to discuss progress and challenges.
			\item Clear Expectations: Set and understand mutual expectations regarding workload and deadlines.
			\item Feedback: Seek and provide constructive feedback to improve efficiency.
		\end{itemize}	
		\item Utilizing University Support Services:
		\begin{itemize}
			\item International Office: Seek assistance with visa issues, accommodation, and integration.
			\item Peer Support: Connect with fellow PhD students to share experiences and advice.
		\end{itemize}			 	
	\end{itemize}
\end{frame}

% new page
\begin{frame}[fragile]{Networking and Professional Development}
\alert{Building a Professional Network}
	\begin{itemize}
		\item Attending Conferences: Participate in national and international academic conferences, present your research to gain visibility and feedback.
		\item Joining Academic Societies: Become a member of relevant academic and professional societies.
		\item Skill Development: Attend workshops on research methodologies, academic writing, and presentation skills.
		\item Training Programs : Participate in training programs offered by the university. Focus on transferable skills such as project management, leadership, and communication.
	
	\end{itemize}
\end{frame}

% new page
\begin{frame}[fragile]{Networking and Professional Development}
\alert{Tips for Effective Networking and Skill Development}
	\begin{itemize}
		\item Be Proactive: Seek out opportunities and take initiative in building your network. Regularly update your LinkedIn profile and academic CV.
		\item Stay Informed: Keep up-to-date with upcoming conferences, workshops, and society events. Follow relevant academic journals and newsletters.
		\item Engage Actively: Participate in discussions and forums both online and offline. Volunteer for roles in academic societies or conference committees.
		\item Follow Up: Maintain relationships by following up with contacts after events.
Use social media and professional platforms to stay connected.

	
	\end{itemize}
\end{frame}

% new page
\begin{frame}[fragile]{Conclusion}
	Summary of Key Points:
Recap of funding sources, financial aspects, practical considerations, and challenges

	Encouragement and Final Thoughts:
Motivational closing, emphasizing the opportunities and support available
\end{frame}


\end{document}
