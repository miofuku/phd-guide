\documentclass[10pt]{beamer}

\usetheme{metropolis}
\usepackage{appendixnumberbeamer}

\usepackage{booktabs}
\usepackage[scale=2]{ccicons}

\usepackage{pgfplots}
\usepgfplotslibrary{dateplot}

\usepackage{xspace}
\newcommand{\themename}{\textbf{\textsc{metropolis}}\xspace}

\title{Understanding the PhD Landscape in Europe and Germany}
\subtitle{Talk 1: Introduction and Getting Started}
\date{\today}
\author{Dr. Bijun Li}
%\institute{Center for modern beamer themes}
% \titlegraphic{\hfill\includegraphics[height=1.5cm]{logo.pdf}}

\begin{document}

\maketitle

%\section{Introduction}

% new page 
\begin{frame}[fragile]{Self-Introduction}


\end{frame}

% new page
\begin{frame}[fragile]{Purpose of the Series}

\alert{Explore the Opportunity of Pursuing a PhD in Europe}
\begin{itemize}
	\item Provide an overview of the PhD landscape in Europe.
	\item Focus on Germany's highly regarded PhD programs.
	\item Provide detailed information on the application process, funding opportunities, and practical considerations specific to Germany.
\end{itemize}

\alert{Target Audience}
\begin{itemize}
	\item Graduate school students currently pursuing a Master's degree and considering a PhD.
	\item Students in the final year of their undergraduate studies exploring future academic opportunities and seeking information on how to prepare for a PhD application in Europe from an early stage.
\end{itemize}
  
\end{frame}

% new page
\begin{frame}[fragile]{Overview}
  %\setbeamertemplate{section in toc}[sections numbered]
  %\tableofcontents[hideallsubsections]
\alert{Content to be Covered:}
  \begin{itemize}
    \item \textbf{Introduction to PhD Programs in Europe}
    \item \textbf{Understanding the German Education System}
    \item \textbf{Finding PhD Opportunities and Supervisors}
    \item \textbf{Application Process and Requirements}
  \end{itemize}
\end{frame}

% new page
\begin{frame}[fragile]{PhD Programs in Europe}
\alert{General Structure}
\begin{itemize}
	\item PhD programs in Europe generally last between 3 to 4 years.
	\item Requirements typically include a Master’s degree or equivalent qualification.
	\item PhD candidates must complete original research culminating in a dissertation or thesis.
	\item Traditional PhD: independent research, minimal coursework, close supervision by a single academic advisor.
	\item Structured PhD: includes formal coursework and seminars, collaboration with research groups, multiple supervisors or a supervisory committee, regular progress reviews and milestones.
\end{itemize}
Comparative Analysis:
Differences between European countries.

\end{frame}

% new page
\begin{frame}[fragile]{PhD Programs in Europe (cont.)}
\alert{Comparative Analysis}
\begin{itemize}
	\item Germany
	\begin{itemize}
		\item Strong focus on research universities and applied sciences.
		\item Well-established industry collaborations.
		\item Typically involves a working contract with a salary.
	\end{itemize}
	\item France
	\begin{itemize}
		\item PhD programs often linked to specific research institutions (CNRS).
		\item Structured programs with defined coursework.
	\end{itemize}
	\item Netherlands
	\begin{itemize}
		\item Known for a structured approach with clear timelines.
		\item Often involves teaching responsibilities.
	\end{itemize}
	\item Scandinavia (e.g., Sweden, Denmark)
	\begin{itemize}
		\item Emphasis on research projects with societal impact.
		\item Programs may include mandatory teaching and public engagement activities.
	\end{itemize}
\end{itemize}

\end{frame}


%\section{Title formats}
% new page
\begin{frame}[fragile]{Overview of the German Higher Education System}
\alert{Research Universities (Universitäten)}
	\begin{itemize}
		\item Focus on academic research and theoretical knowledge.
		\item Known for extensive research opportunities and doctoral programs.
		\item Examples: TU Munich, Heidelberg University, University of Freiburg.
	\end{itemize}
\alert{Universities of Applied Sciences (Fachhochschulen)}
	\begin{itemize}
		\item Strong ties with industry and business sectors.
		\item Offer practice-oriented programs and applied research.
		\item Examples: Munich University of Applied Sciences (Hochschule München), Berlin School of Economics and Law. 
	\end{itemize}
\alert{Research Institutes}
	\begin{itemize}
		\item Dedicated to advanced research in specific fields.
		\item Often operate independently or in affiliation with universities.
		\item Examples: Max Planck Society, Fraunhofer Institutes, Helmholtz Association.
	\end{itemize}
\end{frame}

% new page
\begin{frame}[fragile]{Finding PhD Opportunities}
Researching PhD Programs: University websites, academic journals, and research portals
Contacting Potential Supervisors:
\begin{itemize}
	\item How to identify and approach potential supervisors
	\item Preparing a compelling research proposal
\end{itemize}
Networking:
Attending conferences, workshops, and academic events
	
\end{frame}


% new page
\begin{frame}[fragile]{Application Process and Requirements}
\begin{itemize}
	\item Eligibility Criteria: Academic qualifications (Master’s degree or equivalent)
	\item Application Procedures: Required documents (CV, transcripts, letters of recommendation, research proposal)
	\item Timeline: Key deadlines and milestones in the application process
	\item Language Requirements: Language proficiency tests (e.g., IELTS, TOEFL)
\end{itemize}
\end{frame}

% new page
\begin{frame}[fragile]{Conclusion and Preview of Talk 2}
	Summary of Key Points: Recap of the structure and process of pursuing a PhD in Germany

	Next Talk Preview: Focus on funding, financial aspects, practical considerations, and challenges
\end{frame}


\end{document}
